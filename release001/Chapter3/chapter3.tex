%*****************************************************************************************
%*********************************** Third Chapter **************************************
%*****************************************************************************************
\chapter{Timeline}
    
 \vspace{1mm}
 \footnotesize {
 \begin{flushright}
 \textit{Lo imposible solo tarda un poco m\'as.
%Imagine a world that you would like to see in   10/100/1000  years
 }
%% \cite{auam}.
 \end{flushright}
 }
\vspace{5mm} 

Different tasks have been planned as follow:
\begin{itemize}[noitemsep,topsep=0pt,parsep=0pt,partopsep=0pt]
\item T1: Development of the Low-cost Robot
\item T2: Implement new blocks and tools to Ardublock 
\item T3: Development of the speech recognition tool with pocketsphinx, racket and firmata.
\item T4: Integration of T1, T2 and T3.
\item T5: At the first workshop of Libre Robotics adecuate surveys, methods or plans will 
	  be implemented in order to evaluate mentors and participants. In this part, it 
	  is also going to be designed the activities for the workshop with the aim of 
	  learning and sharing knowledge to build conditions for a better world.
\item T6: Week of the workshop. Evaluation of the first stage and plans for the second stage of the project.
\end{itemize}

Timeline is shown in the gantt chart (Figure \ref{fig:ganttchart}).
Document releases, low-cost robot designs, open-source software tutorials and 
web-page modifications are going to be under constant improvement and 
it is highly probably that new ideas are going to be generated during this process.

% REFERENCE FOR THE GANTT CHART COMMANDS IN LATEX
% http://www.martin-kumm.de/wiki/doku.php?id=Projects:A_LaTeX_package_for_gantt_plots

\begin{figure}[htbp!] 
\begin{center}
  \begin{gantt}{8}{16} %{rows}{columns}
    \begin{ganttitle}
      \numtitle{2014}{1}{2014}{9}
      \numtitle{2015}{1}{2015}{7}
    \end{ganttitle}
    
    \begin{ganttitle}
    \titleelement{April}{1}
    \titleelement{May}{1}
    \titleelement{Jun}{1}
    \titleelement{Jul}{1}
    \titleelement{Aug}{1}
    \titleelement{Sep}{1}
    \titleelement{Oct}{1}
    \titleelement{Nov}{1}
    \titleelement{Dec}{1}
    \titleelement{Jan}{1}
    \titleelement{Feb}{1}
    \titleelement{Mar}{1}
    \titleelement{Apr}{1}
    \titleelement{May}{1}
    \titleelement{Jun}{1}
    \titleelement{Jul}{1}
    \end{ganttitle}
    
  \ganttbar[color=green]{T1}{0}{9}
  \ganttbar[color=green]{T2}{0}{9}
  \ganttbar[color=green]{T3}{2}{7}
  \ganttbar{T4}{8}{2}
  \ganttbar{T5}{9}{6}
  \ganttbar[pattern=crosshatch,color=blue]{T6}{15}{1}
  \end{gantt}
  \caption[PA]{Gantt Chart for {\librER} project}
\label{fig:ganttchart}
  \end{center}
\end{figure}
  
