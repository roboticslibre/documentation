%*****************************************************************************************
%*********************************** Fourth Chapter **************************************
%*****************************************************************************************

\chapter{Future Work}
%----------------------------------------------------------------------

\vspace{1mm}
 \footnotesize {
 \begin{flushright}
 \textit{
 The present is theirs; the future, for which I really worked, is mine.
 }\\
Nikola Tesla 
 \end{flushright}
 }
\vspace{10mm}  

The following points give a brieft review of projects that
can be considered as a future work for {\librER}.
 
%----------------------------------------------------------------------
\section{Inductive Wireless Power System}
%----------------------------------------------------------------------

One of the main requirementes for educational robots is the source power that can mainly 
be covered by either onbaord batteries or an USB cable. However, batteries are 
environmental unfriendly and the USB cable is sometimes an issue for the mobility of the 
robot. Henceforth, in order to get rid of previos-mentioned drawbacks inductive wireless 
power system have proven to be the most suitable solution.
For instance, Deyle \emph{et al.} \cite{DeyleReynolds2008} in 2008 proposed a wireless 
power and bidirectional communication to a swarm of mobile robots in continuous 
operation on a bounded surface of 60cm x 60cm. Deyle's approachs is achieved by 
exciting a large, high \textit{Q} L-C resonator. Deyle compared that cost of lithium 
ion batteries and their design is very favorable. A video of their work is available at 
\url{http://vimeo.com/1900725} which shows the mini-swarm of five battery-free, 
wirelessly powered autonomous mobile robots. In the same fashion, Arunkumar 
\emph{et al.} \cite{arunkumar2010} in 2010 developed an inexpensive, low complexity 
power system capable of providing wireless power from source to sink of 
multiple mobile robots.


%----------------------------------------------------------------------
\section{Building Layout}
%----------------------------------------------------------------------

Recently, Barrett \emph{et al} \cite{Barret2013} tested 751 pupils from 34 
varied classrooms in seven different schools in the UK in which they found 
that environmental factors, namely: colour, choice, connection, complexity, 
flexibility and light, have a significant role to play in influencing academic 
performance. Henceforth, the building enviroment design is one important factor
but mainly a challenging one that few have considered so as to provide 
appropriate facilities to the learners. 
We therefore belive that the enviromental quality where activities
of {\librER} will have been taken place must consider previous-mentioned factors 
to design and build enviroments where participants can discover and develop their
own capabitilies.



% According to the results, 
% once the differences between the “worst” and “best” designed classrooms looked at in the 
% study were taken into account, it was found the be the equivalent to the progress a 
% typical pupil would be expected to make over a year.


% http://www.huffingtonpost.com/2013/01/03/school-design-student-grades_n_2404289.html

% http://www.wired.com/2013/01/school-design-grades/

% http://www.salford.ac.uk/built-environment/about-us/news-and-events/news/study-proves-classroom-design-really-does-matter

% http://www.sciencedirect.com/science/article/pii/S0360132312002582
% A holistic, multi-level analysis identifying the impact of classroom design on pupils’ learning



% http://www.ibi-nightingale.com/environment.html
% n sustainable practices, environmental design is high on our agenda. 



% 
% https://sites.google.com/site/librerobotics2015/facilities-needs
% 
% Facilities Needs
% 
% 
% (4 to 6) desktop computers (programming and website)
% (4) Work Tables at approximately 6' x 2.5' or greater.  Height ideally would be 36" high.
% (16) stools for worktables
% (8) chairs for back counter (programming, CAD and website)
% ? More Internet cables (RJ-45) + cables for any JUHSD computers for the team (one of the cables provided this weekend was cut nearly in half)
% Working Clock
% Looking for donations of these:
% (2) Heavy-Duty extension cords, approximately 25-feet
% Projector:  Looking for a permanent solution.  (We use this every meeting.  incredibly valuable for working in groups and presenting to the entire team.  4 people, for instance, can work on one program or CAD model at the same time by viewing the screen and describing thoughts, code, design to the operator.  The current projector could get reallocated at any time.  Robotics is below the priority of every curriculum need.)
% Push pins for posting robot rules, etc.
% A long list of hand and power tools (we basically have nothing at the moment)
% 




