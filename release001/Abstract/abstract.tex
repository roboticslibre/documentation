%*****************************************************************************************
%*********************************** Abstract ********************************************
%*****************************************************************************************

\begin{abstract}
The aim of this document is to point out the importance of the development of a free 
world-class education in Robotics for kids by proposing, designing and building low-cost 
robots with high standards of quality. On account of Open Source Software and Hardware 
is freely licensed to use, copy, study and change in any way, ten open-source educational 
robots were reviewed. Consequently, it has been proposed a low-cost robot that is based on 
an Arduino UNO board, two hacked continuous servos, and some extra material such as 
recycled CDs, glue gun, etc.; the total cost of the material in local stores is around 
\$  699 Mexican Pesos ($\approx$ \$ 53.86 USD). It has also been 
proposed that Ardublock is going to be the main programming language for the low-cost 
robot considering that it is a programming tool which is very interactive and simple to 
use for inexperienced users. Independently, however, of the low cost of the proposed 
robot, it is robust for basic activities, namely: left and right rotations, movements in 
forward and backward directions to name but a few, and more importantly it is also being 
capable of recognising voice control commands. On the other hand, since {\librER} has 
been thought to be a non-profit organization, people from different fields of study can 
help us to improve the project in different ways by which possible activities and 
open-source software to use have been proposed. As a timeline, a ghantt chart is also 
presented in which six tasks have been considered to be implemented in 16 months: namely;
T1) development of the low-cost robot; T2) implementation of new tools in Ardublock;
T3) development of the speech recognition tool; 
T4) integration of the tasks T1, T2 and T3;
T5) design of the first workshop, and 
T6) evaluation of the complete project.
Finally, as a future work two projects have been briefly reviewed: 1) the implementation 
of inductive wireless power systems to get rid of batteries that are enviromental 
unfriendly and the USB cable that is an issue for the mobility of the robot, and 2) 
the importance of the facilities' enviroment that have a significant role in influencing 
participants for their academic performance and mainly that they can discover and develop 
their own capabitilies.
\end{abstract}
